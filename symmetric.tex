\documentclass[a4paper]{amsart}

\usepackage[euler-digits]{eulervm}
\usepackage{amssymb}
\usepackage[margin=1in]{geometry}
\usepackage{graphicx}

\renewcommand{\thesubsection}{\arabic{subsection}}
\makeatletter
\def\@secnumfont{\bfseries}
\makeatother

\newcommand{\C}{\mathbb{C}}
\newcommand{\Q}{\mathbb{Q}}
\newcommand{\R}{\mathbb{R}}
\renewcommand{\AA}{\mathcal{A}}
\newcommand{\CC}{\mathcal{C}}
\newcommand{\FF}{\mathcal{F}}
\newcommand{\GG}{\mathcal{G}}
\newcommand{\HH}{\mathcal{H}}
\newcommand{\LL}{\mathcal{L}}
\newcommand{\PP}{\mathcal{P}}
\newcommand{\TT}{\mathcal{T}}

\newcommand{\Sym}{\mathsf{Sym}}
\newcommand{\GL}{\mathsf{GL}}
\newcommand{\Hom}{\mathsf{Hom}}
\newcommand{\End}{\mathsf{End}}
\newcommand{\Res}{\mathsf{Res}}
\newcommand{\Ind}{\mathsf{Ind}}
\newcommand{\id}{\mathsf{id}}
\newcommand{\rad}{\mathsf{rad}}
\newcommand{\supp}{\mathsf{supp}}

\newcommand{\GZ}{\mathsf{GZ}}

\newcommand{\Size}[1]{\left|#1\right|}
\newcommand{\Span}[1]{\langle#1\rangle}

\newcommand{\prefix}{\preceq_{\pi}}
\newcommand{\suffix}{\succeq_{\sigma}}

\let\emptyset\varnothing

\begin{document}
\begin{center}
  \textbf{\uppercase{
Gelfand-Tsetlin Algebra.
}}\\
after: Ceccherini-Silberstein - Representation Theory of the Symmetric Groups.
\end{center}

\subsection{Problem}
Classify the (irreducible) matrix representations (over $\C$) of the symmetric group $\Sym_n$.

\subsection{}
Let $G$ be a finite group with \emph{group algebra} $\C G = \{ \sum_{g
  \in G} \alpha_g g : \alpha_g \in \C \}$.  An element $a = \sum_{g
  \in G} \alpha_g g \in \C G$ can be regarded as a function $a \colon G
\to \C$, $g \mapsto \alpha_g$.  From 
\begin{align*}
  a a' = \sum_{g \in G} \alpha_g g \sum_{g' \in G} \alpha'_g g'
= \sum_{g \in G} \sum_{g' \in G} \alpha_g  \alpha'_g  g g'
= \sum_{h \in G}  (\sum_{gg' = h} \alpha_g \alpha'_{g'}) h
= \sum_{h \in G}  (\sum_g \alpha_g \alpha'_{g^{-1}h}) h,
\end{align*}
the \emph{convolution} product  on $\C G$ is given by $(a \star a')(h) = 
\sum_g a(g) a'(g^{-1}h)$.

\subsection{}
A \emph{representation} of
$G$ is a homomorphism $\rho \colon G \to \GL(V)$ for some
(finite-dimensional) $\C$-space $V$.  This extends to a unique
homomorphism $\rho \colon \C G \to \End(V)$. Then $V$ is a \emph{$\C
  G$-module} (or simply a \emph{$G$-module}) with linear action $(v,
a) \mapsto v.a= v \rho(a)$, $v \in V$, $a \in \C G$.

\subsection{Schur's Lemma 1.2.1}
Let $\hat{G}$ be a complete set of irreducible representations $ \rho
\colon G \to \GL(W^{\rho})$ such that $\{W^{\rho}: \rho \in \hat{G}\}$
are the simple $G$-modules.  Let $\rho, \sigma \in \hat{G}$.  Then
$\Hom_G(W^{\rho}, W^{\sigma}) = \{0\}$ unless $\rho = \sigma$, and
$\End_G(W^{\rho}) = \Hom_G(W^{\rho}, W^{\rho}) =
\Span{\id_{W^{\rho}}}$ is $1$-dimensional as $\C$-space.


\subsection{}
Let $V = \C^n$ be a $G$-module.  By complete reducibility, there are
certain \emph{multiplicities} $m_{\rho}$, $\rho \in \hat{G}$, such that, up
to isomorphism, $V = \bigoplus_{\rho \in \hat{G}} m_{\rho} W^{\rho} =
\bigoplus_{\rho \in \hat{G}} \bigoplus_{j = 1}^{m_{\rho}} W^{\rho}_j$.  Let
$I^{\rho}_j \colon W^{\rho}_j \to V$ and $E^{\rho}_j \colon V \to
W^{\rho}_j$ be the natural injection of and the natural projection
onto the component $W^{\rho}_j$.  Then $\id_V = 
= \sum_{\rho} \sum_{j=1}^{m_{\rho}} I^{\rho}_j =
\sum_{\rho} \sum_{j=1}^{m_\rho} E^{\rho}_j$.


% \subsection{}
% $\{I^{\rho}_j : j = 1, \dots, m_{\rho} \}$ is a basis of $\Hom_G(W^{\rho}, V)$,
% whence $\dim \Hom_G(W^{\rho}, V) = m_{\rho}$.

\subsection{Thm 1.2.14}
$\End_G V = \Hom_G(V, V) = \bigoplus_{\rho} M_{m_{\rho}, m_{\rho}}(\C)$
whence $\dim \End_G V = \sum_{\rho} m_{\rho}^2$.

\subsection{Proof}
 Let $\phi \in \End_G V$.  Then $\phi
 = \id_V \circ \phi \circ \id_V
 = (\sum_{\rho,j} E^{\rho}_j) \circ \phi \circ (\sum_{\sigma,k} I^{\sigma}_k)
 = \sum_{\rho,j,\sigma,k} E^{\rho}_j \circ \phi \circ I^{\sigma}_k
 = \sum_{\rho,j,k} a_{jk} E^{\rho}_j \circ E^{\rho}_k$, for certain $a_{jk}$, by Schur's Lemma.


\subsection{Def 1.2.16}
A $G$-module $V = \bigoplus_{\rho} m_{\rho} W^{\rho}$ is
\emph{multiplicity-free} if $m_{\rho} < 2$ for all $\rho \in \hat{G}$.

\subsection{Cor 1.2.17}
$V$ is multiplicity-free $\iff \End_G(V)$ is commutative.


\subsection{Permutation Representations}
Let $X$ be a $G$-set, with corresponding permutation module $V = \C X$.
The action of $G$ on $X$ induces a $G$-action on $X \times X$.

\subsection{Prop 1.4.1}
$\End(\C X) = \C (X \times X)$ with product $(f \ast g)(x, y) =
\sum_{z \in X} f(x, z) g(z, y)$ ($X \times X$-matrices over $\C$).
And $\End_G(\C X) = \C (X \times X)^G$, the functions that are
constant on $G$-orbits.

\subsection{}
For $(x, y) \in X \times X$ denote $(x, y)^{\dagger}:= (y, x)$.
A $G$-orbit $\Omega \subseteq X \times X$ is \emph{symmetric}
if $\Omega^{\dagger} = \Omega$.  The $G$-set $X$ is \emph{symmetric} if
all $G$-orbits on $X \times X$ are symmetric.


\subsection{Gelfand's Lemma, symmetric case 1.4.8}
If  $X$ is a symmetric $G$-set  then $\End_G(\C X)$ is commutative.

\subsection{Proof}
Let $f, g \in \C (X \times X)^G$, and regard them as functions
$f, g \colon X \times X \to \C$ that are constant on $G$-orbits.
Then $(f \ast g)(x, y) 
= \sum_{z \in X} f(x, z) g(z, y)
= \sum_{z \in X} g(y, z) f(z, x)
= (g \ast f)(x, y)$.

\subsection{Gelfand Pairs 1.4.9}
Let $K \leq G$.  Then $(G, K)$ is a \emph{Gelfand Pair} if the
permutation representation of $G$ on $X = G/K$ is multiplicity-free.
If $X$ is a symmetric $G$-set then $(G,K)$ is a \emph{symmetric
  Gelfand pair}.

\subsection{Exp 1.4.10}
Let $X = \{1, \dots,n\}$.  Then, for $0 \leq k \leq n$, 
$\Sym_n$ acts transitively on $\binom{X}{k} = \{A \subseteq X : \Size{A} = k\}$
with stabilizer isomorphic to $\Sym_k \times \Sym_{n-k}$.
Note that $(A, B)$ and $(A', B')$ in $\binom{X}{k} \times \binom{X}{k}$
are in the same $G$-orbit if and only if $\Size{A \cap B} = \Size{A' \cap B'}$.
Each orbit is symmetric since $\Size{A \cap B}  = \Size{B \cap A}$.


\subsection{Thm 1.4.12}
(Frobenius Reciprocity.) Let $K \leq G$ and let $X = G/K$. Suppose $V
=\C X = \sum_{\rho} m_{\rho} W^{\rho}$.  Then $m_{\rho} = \dim
(W^{\rho})^K$ for all $\rho \in \hat{G}$.

\subsection{Cor 1.4.13}
$(G, K)$ is a Gelfand pair if and only if $\dim (W^{\rho})^K \leq 1$
for all $\rho \in \hat{G}$.


\subsection{}
If $X = G/K$ then $\C(X \times X)^G \cong \C X^K \cong \C (K\backslash
G/K)$.  Thus $(G,K)$ is a Gelfand pair if and only if $\C (K\backslash
G/K)$ (as subalgebra of $\C G$) is commutative.  Moreover, $X$ is
symmetric if and only if $g^{-1} \in KgK$ for all $g\in G$.

\subsection{Gelfand's Lemma}
Let $K \leq G$ and suppose $\tau \colon G \to G$ is an automorphism
such that $g^{-1} \in K \tau(g) K$ for all $g \in G$.  Then $(G, K)$ is
a Gelfand pair.

\subsection{Proof} ...



\subsection{Exp 1.5.26}
$(G \times G, \tilde{G})$ is a Gelfand pair.
$G \times G$ acts on $X = G$ as $g.(a,b) = a^{-1} g b$.
The stabilizer of $1 \in G$ is the diagonal subgroup $\tilde{G} = \{(g, g) : g \in G\}$ of $G \times G$
Using the automorphism $\tau \colon (a, b) \mapsto (b, a)$ of $G \times G$, we get $(a, b)^{-1} = (a^{-1}, b^{-1}) = 
(a^{-1}, a^{-1})(b, a)(b^{-1}, b^{-1}) \in 
\tilde{G}\tau(a,b)\tilde{G}$.


\subsection{Conjugacy invariant functions}
Regard $f = \sum_{g \in G} \alpha_g g \in \C G$ as a function
$f \colon G \to \C$, $f(g) = \alpha_g$.
Let $H \leq G$ and set 
$\CC(G, H) = \{ f \in \C G : f(g) = f(g^h) \text{ for all } g \in G,\, h \in H\}$.

\subsection{Lem 2.1.1}
$\CC(G, H)$ is commutative if and only if $(H \times G, \tilde{H})$ is
a Gelfand pair.  Since $\CC(G, H) = \End_{H \times G}(\C X)$, where $X
= (H \times G)/\tilde{H} = G$ with action $\eta:$ $a.\eta(h,g) = h^{-1}a g$.


\subsection{Cor 2.1.4}
$\eta = \sum_{\sigma \in \hat{G}, \rho \in \hat{H}} m_{\rho, \sigma} (\sigma \times \rho)$, where 
$m_{\rho, \sigma} = \dim \Hom_H(\rho, \Res^G_H \sigma')$,
and where $\sigma'$ is the adjoint of $\sigma$: $\sigma'(g) = \sigma(g^{-1})^{\mathsf{tr}}$.



\subsection{Multiplicity-free subgroups 2.1.9}
A subgroup $H$ of $G$ is \emph{multiplicity-free} if
$\Res^G_H \rho$ is a multiplicity-free representation of $H$ for all 
$\rho \in \hat{G}$.


\subsection{Thm 2.1.10} $H \leq G$ is multiplicity-free if and only if
$(H \times G, \tilde{H})$ is
a Gelfand pair.


\subsection{Pro 2.1.12} 
$(G \times H, \tilde{H})$ is a symmetric Gelfand pair if and only if 
for each $g \in G$ there exists $h \in H$ such that $g^{-1} = g^h$
(i.e., if every $g \in G$ is $H$-conjugate to its inverse).

\subsection{Multiplicity-free chains.}
A chain of subgroups 
\begin{align*} \label{eq:chain} \tag{$*$}
1 = G_0 < G_1 < \dots < G_n = G
\end{align*}
is
\emph{multiplicity-free} if $G_{i-1}$ is a multiplicity-free subgroup
of $G_i$ for  $1 \leq i \leq n$.
The \emph{branching graph} of  a multiplicity-free chain~\eqref{eq:chain} is the directed graph with  vertex set
$\coprod_{i=0}^n \hat{G}_i$
and edge set $\{(\rho, \sigma) \in \hat{G}_i \times \hat{G}_{i-1} : \sigma \preceq \Res^{G_i}_{G_{i-1}} \rho,\, 1  \leq i \leq n \}$.  Write $\rho \to \sigma$ for the edge $(\rho, \sigma)$.
For an irreducible representation $\rho_n \in \hat{G}_n$, set $T(\rho_n) = \{ (\rho_n \to \rho_{n-1} \to \dots \to \rho_1 \to \rho_0) : \rho_i \in \hat{G}_i\}$,
the set of all paths in the branching graph from $\rho_n$ to the
trivial representation $\rho_0$ of the trivial group $G_0$.

\subsection{Gelfand-Tsetlin basis}

Restricting $\rho_n$ along the chain~\eqref{eq:chain} and decomposing into irreducible
constituents at each step, one gets
\[
W^{\rho_n} = \bigoplus_{\rho_n\to \rho_{n-1}} W^{\rho_{n-1}}
= \bigoplus_{\rho_n\to \rho_{n-1}} \bigoplus_{\rho_{n-1} \to \rho_{n-2}} W^{\rho_{n-2}}
= \dots
= \bigoplus_{T\in T(\rho_n)} W^{\rho_0},
\]
a decomposition of the space $W^{\rho_n}$ into uniquely determined
orthogonal $1$-dimensional subspaces.  A \emph{Gelfand-Tsetlin basis}
of $W^{\rho_n}$ is obtained by choosing, for each $T \in T(n)$, a norm
$1$ vector $v_T$ in the corresponding line $W^{\rho_0}$.

\subsection{Exp} $1 < C_3 < A_4$.

\subsection{Gelfand-Tsetlin algebra}

Denote by $Z(i)$ the center of the group algebra $\C G_i$ in the
chain~\eqref{eq:chain}.  The \emph{Gelfand-Tsetlin algebra} $\GZ(n)$ of a
multiplicity-free chain is the subalgebra of $\C G$ generated by the
$Z(i)$, $i = 0, \dots, n$.

\subsection{Thm 2.2.2}
$\GZ(n)$ is a maximal abelian subalgebra of $\C G$,
consisting of those elements $a \in \C G$ which
are diagonlized by the Gelfand-Tsetlin basis $T(\rho)$ of
$W^{\rho}$ for each $\rho \in \hat{G}$.

\subsection{Cor 2.2.3}
For $\rho \in \hat{G}$, every basis element $v_T$, $T \in \TT(\rho)$,
is a common eigenvector of the matrices $\rho(f)$, $f \in \GZ(n)$.


\subsection{Pro 2.2.7}
If $G_0 < \dots < G_n$ is a multiplicity-free chain then
$\CC(G_n, G_{n-1}) \subseteq \GZ(n)$.




\end{document}
