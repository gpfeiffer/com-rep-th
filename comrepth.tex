%%%%%%%%%%%%%%%%%%%%%%%%%%%%%%%%%%%%%%%%%%%%%%%%%%%%%%%%%%%%%%%%%%%%%%%%%%%%%
%%
%%  comrepth.tex
%%
\documentclass[12pt]{amsart}

\title{Computational Representation Theory}

\usepackage[euler-digits]{eulervm}
\usepackage[a4paper,margin=1in,includeheadfoot]{geometry}
\usepackage{amssymb}

\newtheorem{theorem}{Theorem}[section]
\newtheorem{proposition}[theorem]{Proposition}
\newtheorem{lemma}[theorem]{Lemma}
\newtheorem{corollary}[theorem]{Corollary}

\theoremstyle{definition}
\newtheorem{example}[theorem]{Example}
\newtheorem{definition}[theorem]{Definition}
\newtheorem{remark}[theorem]{Remark}

\renewcommand{\emptyset}{\varnothing}

%%%%%%%%%%%%%%%%%%%%%%%%%%%%%%%%%%%%%%%%%%%%%%%%%%%%%%%%%%%%%%%%%%%%%%%%%%%%%
\begin{document}

\maketitle

\tableofcontents

%%%%%%%%%%%%%%%%%%%%%%%%%%%%%%%%%%%%%%%%%%%%%%%%%%%%%%%%%%%%%%%%%%%%%%%%%%%%%
\section{Introduction}
\label{sec:intro}

Representation theory of finite groups is concerned with the ways of writing a group $G$ as a group of automorphisms and being able to analyze the group in terms of these automorphisms.\\
If $X$ is a structure then we have End$(X)=\{f:X\rightarrow X$, $f$ compatible with structure$\}$.  The set of automorphisms is given by Aut$(X)=\{f\in$End$(X)$, $f$ invertible$\}$.  Aut$(X)$ is a groups under composition, as it satisfies the group axioms:
\begin{itemize}
\item compositions of automorphisms are automorphisms; \item composition of maps is associative; \item id$_X$ is the identity; \item $f\in$ Aut$(X)$ is invertible by definition.
\end{itemize}
A representation realises a group $G$ as a group of automorphisms of some structure $X$.  It associates to each element $a\in G$ an automorphism of $X$ in a way that is compatible with the multiplication in $G$.

\begin{definition}
A \emph{representation} of a group $G$ is a homomorphism $\phi:G\rightarrow$Aut$(X)$ for some structure $X$.  For any representation we have
\begin{itemize}
\item $\phi(a^{-1})=\phi(a)^{-1}$; \item $\phi(1)=$id$_X$; \item $\phi(ab)=\phi(a)\phi(b)$.
\end{itemize}
\end{definition}

\begin{example}
Let $X=\{1,\ldots,n\}$ where $n\in\mathbb{N}$, ie. $X$ is a finite set.  Then End$(X)=\{f:x\rightarrow X\}=X^X=X^n$ and $|$End$(X)|=n^n$.  We have Aut$(X)=$Sym$(X)=$Sym$_n$, the symmetric group whose elements are all the permutations of $X$, with $|$Aut$(X)|=n!$.  The permutation representation is $\phi:G\rightarrow$Sym$(X)$.
\end{example}

\begin{example}
Let $X=\mathbb{F}^n$, a vector space over a field $\mathbb{F}$.  Then End$(X)=\mathbb{F}^{n\times n}$ and if $|\mathbb{F}|=q$ then $|$End$(X)|=q^{n^2}$.  We have Aut$(X)=GL(X)=GL_n(\mathbb{F})$, the general linear group of all invertible matrices over $\mathbb{F}$.  In this case,
\begin{align*}
|GL_n(\mathbb{F})| &= (q^n-1)(q^n-q)(q^n-q^2)\ldots(q^n-q^{n-1})\\
&= q^{n\choose 2}\prod_{i=1}^n(q^i-1).
\end{align*}
The matrix representation is $\phi:G\rightarrow GL_n(\mathbb{F})$.
\end{example}

Finding \textbf{one} representation of a group gives a concrete description of the group, allowing us to perform explicit calculations with its elements.  Classifying \textbf{all} representations provides structural information about the group and yields general results which are otherwise very difficult to prove.

\begin{example}
Let $X = G$. $Aut(G) =$ all invertible group homomorphisms $f: G \rightarrow G$.
\end{example}

\begin{example}
$G = \mathbb{Z}_{n} = \mathbb{Z} / n \mathbb{Z} = \left\{ 0, \ldots , n - 1 \right\}$
$+_{n}$ and $\cdot_{n}$ are taken modulo n: 
$a +_{n} b := ( a + b ) mod n$ for $a, b \in \mathbb{Z}_{n}$
$\left( \mathbb{Z}_{n}, +_{n} \right)$ is a commutative group. We want to find
$End(X)$
Thus we need to find all homomorphisms $f: \mathbb{Z}_{n} \rightarrow \mathbb{Z}_{n}$.


$$f(0) = 0$$
$$a \in \mathbb{Z}_{n} : a = \underbrace{1 + \ldots + 1}_{a \text{ times}}$$
$$f(a) = f( \underbrace{1 + \ldots + 1}_{a \text{ times}})$$
$$ = \underbrace{f(1) + \ldots + f(1)}_{a \text{ times}}$$

$f$ is determined by choosing $f(1)$, any choice in $\mathbb{Z}_{n}$ is possible. Then $f$ is multiplication by $f(1)$.

$$End(X) = \left\{ X \mapsto X \cdot c : c \in \mathbb{Z}_{n} \right\}$$
$$Aut(X) = \mathbb{Z}_{n}^{\ast} = \left\{ c \in \mathbb{Z}_{n} : gcd ( c, n ) = 1 \right\}$$
\end{example} 

\begin{example}
The automorphisms of a permutation.
$X = \left\{ 1, \ldots, n \right\}, R \subseteq X^{2}$ is a binary relation (structure on $X$).
$$End(X, R) = \left\{ f: X \rightarrow X, x R y \Rightarrow f(x) R f(y) \right\}$$
$$Aut(X) = \left\{ \text{permutations in } End(X) \right\}$$

A permutation of $X$ is a binary relation:
\begin{itemize}
\item As a function $x \mapsto x^{\alpha}$:
$$\rightarrow Aut( \alpha ) = \left\{\text{ individual cycles, swap cycles of the same length} \right\}.$$
\noindent Different ways of writing $\alpha$ in a prescribed cycle shape ($ \text{Centralizer of } \alpha \in Sym_{n}$).
\item Total order on $X \left( \text{ write $\alpha$ as an image list}: \begin{pmatrix}1^{\alpha} & 2^{\alpha} & \ldots & n^{\alpha} \end{pmatrix} \rightarrow Aut(X) = 1 \right).$
\end{itemize}
\end{example}

\subsection{Groups}
\begin{definition}
Recall: a set $G$ together with a binary operation $\ast: G \times G \rightarrow G$, $\left( a, b \right) \mapsto a \ast b$ is a group if:
\begin{enumerate}
\item $\left( a \ast b \right) \ast c = a \ast \left( b \ast c \right)$ (Associativity).
\item There exists an element $e \in G$ such that $e \ast a = a = a \ast e$ for all $a \in G$ (identity).
\item For each $a \in G$ there is an element $a^{ - 1} \in G$ such that $a \ast a^{- 1} = a^{- 1} \ast a = e$ (inverse).
\end{enumerate}
\end{definition}

\begin{remark}
\begin{itemize}
\item Usually $e$ is written $1_{G}$ or 1.
\item $\left( G, \ast \right)$ is a monoid if it satisfies (1) and (2) above.
\item $End(X)$ is a monoid.
\item Let $M$ be a monoid then, $M^{\sharp} = \left\{ a \in M : a \text{ invertible} \right\}$ is a group.
\end{itemize}
\end{remark}

\begin{example}(Examples of Groups).
\begin{enumerate}
\item $\zeta_{n} = e^{2 \pi i / 2}$, a primitive $n^{th}$ root of unity. 
$C_{n} = \left\{ \zeta_{n}^{i} : i \in \mathbb{Z} \right\}$ is a cyclic group. 
$$\left( C_{n}, \cdot \right) \longleftrightarrow \left( \mathbb{Z}_{n}, + \right)$$
$$\zeta_{n}^{k} \longleftrightarrow k \text{ mod } n$$
\item$\left( \mathbb{Z}, + \right)$, the infinite cyclic group.
\item Rotations and reflections of regular $n$-gons: dihedral groups.
\item Symmetric groups and general linear groups.
\end{enumerate}
\end{example}

\nocite{*}

%%%%%%%%%%%%%%%%%%%%%%%%%%%%%%%%%%%%%%%%%%%%%%%%%%%%%%%%%%%%%%%%%%%%%%%%%%%%%
\bibliographystyle{amsplain}
\bibliography{comrepth}

%%%%%%%%%%%%%%%%%%%%%%%%%%%%%%%%%%%%%%%%%%%%%%%%%%%%%%%%%%%%%%%%%%%%%%%%%%%%%
\end{document}
