%%%%%%%%%%%%%%%%%%%%%%%%%%%%%%%%%%%%%%%%%%%%%%%%%%%%%%%%%%%%%%%%%%%%%%%%%%%%%
%%
%%  comrepth.tex
%%
\documentclass[12pt]{amsart}

\title{Computational Representation Theory}

\usepackage[euler-digits]{eulervm}
\usepackage[a4paper,margin=1in,includeheadfoot]{geometry}
\usepackage{amssymb}

\newtheorem{theorem}{Theorem}[section]
\newtheorem{proposition}[theorem]{Proposition}
\newtheorem{lemma}[theorem]{Lemma}
\newtheorem{corollary}[theorem]{Corollary}

\theoremstyle{definition}
\newtheorem{example}[theorem]{Example}
\newtheorem{definition}[theorem]{Definition}
\newtheorem{remark}[theorem]{Remark}

\renewcommand{\emptyset}{\varnothing}

%%%%%%%%%%%%%%%%%%%%%%%%%%%%%%%%%%%%%%%%%%%%%%%%%%%%%%%%%%%%%%%%%%%%%%%%%%%%%
\begin{document}

\maketitle

\tableofcontents

%%%%%%%%%%%%%%%%%%%%%%%%%%%%%%%%%%%%%%%%%%%%%%%%%%%%%%%%%%%%%%%%%%%%%%%%%%%%%
\section{Introduction}
\label{sec:intro}

Representation theory of finite groups is concerned with the ways of writing a group $G$ as a group of automorphisms and being able to analyze the group in terms of these automorphisms.\\
If $X$ is a structure then we have End$(X)=\{f:X\rightarrow X$, $f$ compatible with structure$\}$.  The set of automorphisms is given by Aut$(X)=\{f\in$End$(X)$, $f$ invertible$\}$.  Aut$(X)$ is a groups under composition, as it satisfies the group axioms:
\begin{itemize}
\item compositions of automorphisms are automorphisms; \item composition of maps is associative; \item id$_X$ is the identity; \item $f\in$ Aut$(X)$ is invertible by definition.
\end{itemize}
A representation realises a group $G$ as a group of automorphisms of some structure $X$.  It associates to each element $a\in G$ an automorphism of $X$ in a way that is compatible with the multiplication in $G$.

\begin{definition}
A \emph{representation} of a group $G$ is a homomorphism $\phi:G\rightarrow$Aut$(X)$ for some structure $X$.  For any representation we have
\begin{itemize}
\item $\phi(a^{-1})=\phi(a)^{-1}$; \item $\phi(1)=$id$_X$; \item $\phi(ab)=\phi(a)\phi(b)$.
\end{itemize}
\end{definition}

\begin{example}
Let $X=\{1,\ldots,n\}$ where $n\in\mathbb{N}$, ie. $X$ is a finite set.  Then End$(X)=\{f:x\rightarrow X\}=X^X=X^n$ and $|$End$(X)|=n^n$.  We have Aut$(X)=$Sym$(X)=$Sym$_n$, the symmetric group whose elements are all the permutations of $X$, with $|$Aut$(X)|=n!$.  The permutation representation is $\phi:G\rightarrow$Sym$(X)$.
\end{example}

\begin{example}
Let $X=\mathbb{F}^n$, a vector space over a field $\mathbb{F}$.  Then End$(X)=\mathbb{F}^{n\times n}$ and if $|\mathbb{F}|=q$ then $|$End$(X)|=q^{n^2}$.  We have Aut$(X)=GL(X)=GL_n(\mathbb{F})$, the general linear group of all invertible matrices over $\mathbb{F}$.  In this case,
\begin{align*}
|GL_n(\mathbb{F})| &= (q^n-1)(q^n-q)(q^n-q^2)\ldots(q^n-q^{n-1})\\
&= q^{n\choose 2}\prod_{i=1}^n(q^i-1).
\end{align*}
The matrix representation is $\phi:G\rightarrow GL_n(\mathbb{F})$.
\end{example}

Finding \textbf{one} representation of a group gives a concrete description of the group, allowing us to perform explicit calculations with its elements.  Classifying \textbf{all} representations provides structural information about the group and yields general results which are otherwise very difficult to prove.

\nocite{*}

%%%%%%%%%%%%%%%%%%%%%%%%%%%%%%%%%%%%%%%%%%%%%%%%%%%%%%%%%%%%%%%%%%%%%%%%%%%%%
\bibliographystyle{amsplain}
\bibliography{comrepth}

%%%%%%%%%%%%%%%%%%%%%%%%%%%%%%%%%%%%%%%%%%%%%%%%%%%%%%%%%%%%%%%%%%%%%%%%%%%%%
\end{document}
