%%%%%%%%%%%%%%%%%%%%%%%%%%%%%%%%%%%%%%%%%%%%%%%%%%%%%%%%%%%%%%%%%%%%%%%%%%%%%
%%
%%  comrepth.tex
%%
\documentclass[12pt]{amsart}

\title{Computational Representation Theory}

\usepackage[euler-digits]{eulervm}
\usepackage[a4paper,margin=1in,includeheadfoot]{geometry}
\usepackage{amssymb}

\newtheorem{theorem}{Theorem}[section]
\newtheorem{proposition}[theorem]{Proposition}
\newtheorem{lemma}[theorem]{Lemma}
\newtheorem{corollary}[theorem]{Corollary}

\theoremstyle{definition}
\newtheorem{example}[theorem]{Example}
\newtheorem{definition}[theorem]{Definition}
\newtheorem{remark}[theorem]{Remark}

\renewcommand{\emptyset}{\varnothing}

%%%%%%%%%%%%%%%%%%%%%%%%%%%%%%%%%%%%%%%%%%%%%%%%%%%%%%%%%%%%%%%%%%%%%%%%%%%%%
\begin{document}

\maketitle

\tableofcontents

%%%%%%%%%%%%%%%%%%%%%%%%%%%%%%%%%%%%%%%%%%%%%%%%%%%%%%%%%%%%%%%%%%%%%%%%%%%%%
\section{Introduction}
\label{sec:intro}

Representation theory of finite groups is concerned with the ways of writing a group $G$ as a group of automorphisms and being able to analyze the group in terms of these automorphisms.\\
If $X$ is a structure then we have $End(X)=\{f:X\rightarrow X$, $f$ compatible with structure$\}$.  The set of automorphisms is given by $Aut(X)=\{f\in End(X)$, $f$ invertible$\}$.  $Aut(X)$ is a groups under composition, as it satisfies the group axioms:
\begin{itemize}
\item compositions of automorphisms are automorphisms; \item composition of maps is associative; \item id$_X$ is the identity; \item $f\in$ Aut$(X)$ is invertible by definition.
\end{itemize}
A representation realises a group $G$ as a group of automorphisms of some structure $X$.  It associates to each element $a\in G$ an automorphism of $X$ in a way that is compatible with the multiplication in $G$.

\begin{definition}
A \emph{representation} of a group $G$ is a homomorphism $\phi:G\rightarrow Aut(X)$ for some structure $X$.  For any representation we have
\begin{itemize}
\item $\phi(a^{-1})=\phi(a)^{-1}$; \item $\phi(1)=id_X$; \item $\phi(ab)=\phi(a)\phi(b)$.
\end{itemize}
\end{definition}

\begin{example}
Let $X=\{1,\ldots,n\}$ where $n\in\mathbb{N}$, ie. $X$ is a finite set.  Then $End(X)=\{f:x\rightarrow X\}=X^X=X^n$ and $|End(X)|=n^n$.  We have $Aut(X)=Sym(X)=Sym_n$, the symmetric group whose elements are all the permutations of $X$, with $|Aut(X)|=n!$.  The permutation representation is $\phi:G\rightarrow Sym(X)$.
\end{example}

\begin{example}
Let $X=\mathbb{F}^n$, a vector space over a field $\mathbb{F}$.  Then $End(X)=\mathbb{F}^{n\times n}$ and if $|\mathbb{F}|=q$ then $|End(X)|=q^{n^2}$.  We have Aut$(X)=GL(X)=GL_n(\mathbb{F})$, the general linear group of all invertible matrices over $\mathbb{F}$.  In this case,
\begin{align*}
|GL_n(\mathbb{F})| &= (q^n-1)(q^n-q)(q^n-q^2)\ldots(q^n-q^{n-1})\\
&= q^{n\choose 2}\prod_{i=1}^n(q^i-1).
\end{align*}
The matrix representation is $\phi:G\rightarrow GL_n(\mathbb{F})$.
\end{example}

Finding \textbf{one} representation of a group gives a concrete description of the group, allowing us to perform explicit calculations with its elements.  Classifying \textbf{all} representations provides structural information about the group and yields general results which are otherwise very difficult to prove.

\begin{example}
Let $X = G$. $Aut(G) =$ all invertible group homomorphisms $f: G \rightarrow G$.
\end{example}

\begin{example}
$G = \mathbb{Z}_{n} = \mathbb{Z} / n \mathbb{Z} = \left\{ 0, \ldots , n - 1 \right\}$
$+_{n}$ and $\cdot_{n}$ are taken modulo n:
$a +_{n} b := ( a + b ) mod n$ for $a, b \in \mathbb{Z}_{n}$
$\left( \mathbb{Z}_{n}, +_{n} \right)$ is a commutative group. We want to find
$End(X)$
Thus we need to find all homomorphisms $f: \mathbb{Z}_{n} \rightarrow \mathbb{Z}_{n}$.


$$f(0) = 0$$
$$a \in \mathbb{Z}_{n} : a = \underbrace{1 + \ldots + 1}_{a \text{ times}}$$
$$f(a) = f( \underbrace{1 + \ldots + 1}_{a \text{ times}})$$
$$ = \underbrace{f(1) + \ldots + f(1)}_{a \text{ times}}$$

$f$ is determined by choosing $f(1)$, any choice in $\mathbb{Z}_{n}$ is possible. Then $f$ is multiplication by $f(1)$.

$$End(X) = \left\{ X \mapsto X \cdot c : c \in \mathbb{Z}_{n} \right\}$$
$$Aut(X) = \mathbb{Z}_{n}^{\ast} = \left\{ c \in \mathbb{Z}_{n} : gcd ( c, n ) = 1 \right\}$$
\end{example}

\begin{example}
The automorphisms of a permutation.
$X = \left\{ 1, \ldots, n \right\}, R \subseteq X^{2}$ is a binary relation (structure on $X$).
$$End(X, R) = \left\{ f: X \rightarrow X, x R y \Rightarrow f(x) R f(y) \right\}$$
$$Aut(X) = \left\{ \text{permutations in } End(X) \right\}$$

A permutation of $X$ is a binary relation:
\begin{itemize}
\item As a function $x \mapsto x^{\alpha}$:
$$\rightarrow Aut( \alpha ) = \left\{\text{ individual cycles, swap cycles of the same length} \right\}.$$
\noindent Different ways of writing $\alpha$ in a prescribed cycle shape ($ \text{Centralizer of } \alpha \in Sym_{n}$).
\item Total order on $X \left( \text{ write $\alpha$ as an image list}: \begin{pmatrix}1^{\alpha} & 2^{\alpha} & \ldots & n^{\alpha} \end{pmatrix} \rightarrow Aut(X) = 1 \right).$
\end{itemize}
\end{example}

\subsection{Groups}
\begin{definition}
Recall: a set $G$ together with a binary operation $\ast: G \times G \rightarrow G$, $\left( a, b \right) \mapsto a \ast b$ is a group if:
\begin{enumerate}
\item $\left( a \ast b \right) \ast c = a \ast \left( b \ast c \right)$ (Associativity).
\item There exists an element $e \in G$ such that $e \ast a = a = a \ast e$ for all $a \in G$ (identity).
\item For each $a \in G$ there is an element $a^{ - 1} \in G$ such that $a \ast a^{- 1} = a^{- 1} \ast a = e$ (inverse).
\end{enumerate}
\end{definition}

\begin{remark}
\begin{itemize}
\item Usually $e$ is written $1_{G}$ or 1.
\item $\left( G, \ast \right)$ is a monoid if it satisfies (1) and (2) above.
\item $End(X)$ is a monoid.
\item Let $M$ be a monoid then, $M^{\sharp} = \left\{ a \in M : a \text{ invertible} \right\}$ is a group.
\end{itemize}
\end{remark}

\begin{example}(Examples of Groups).
\begin{enumerate}
\item $\zeta_{n} = e^{2 \pi i / 2}$, a primitive $n^{th}$ root of unity.
$C_{n} = \left\{ \zeta_{n}^{i} : i \in \mathbb{Z} \right\}$ is a cyclic group.
$$\left( C_{n}, \cdot \right) \longleftrightarrow \left( \mathbb{Z}_{n}, + \right)$$
$$\zeta_{n}^{k} \longleftrightarrow k \text{ mod } n$$
\item$\left( \mathbb{Z}, + \right)$, the infinite cyclic group.
\item Rotations and reflections of regular $n$-gons: dihedral groups.
\item Symmetric groups and general linear groups.
\end{enumerate}
\end{example}

\section{$G$-sets}
\label{sec:G-sets}
Let $G$ be a finite group.  The idea of this chapter is to study \textbf{actions} of $G$ on sets $X$.  We will see that this is a (more) convenient way to look at permutation representations.  (\textbf{Recall:} a permutation representation of $G$ is a homomorphism $\phi:G\rightarrow Sym(X)$.)

\begin{definition}
Let $X$ be a finite set.  An \emph{action} of $G$ on $X$ is a map $X\times G\rightarrow X$, $(x, a)\mapsto x.a$ such that:
\begin{description}
\item[(A1)] $(x.a).b=x.(ab)$ for all $x\in X$, $a, b\in G$ (Composition);
\item[(A2)] $x.1=x$ for all $x\in X$ (Identity).
\end{description}
\end{definition}

\textbf{Note:} $(A1)\nRightarrow(A2).$  For example, if $X=\{0, 1\}$ with $x.a=0$ for all $x\in X$ and $a\in G$, this satisfies $(A1)$ but not $(A2)$.

\begin{example}
We give some examples of actions.
\begin{enumerate}
\item $Sym(X)$ acts "naturally" on $X$.  We write $x^a$ for the image of $x\in X$ under $a\in Sym(X)$.  Then we can define an action of $Sym(X)$ on $X$ as $x.a:=x^a$.  The product $ab$ of two permutations $a, b\in Sym(X)$ is the result of first applying $a$ and then applying $b$.  By $(A1)$ we have $x^{ab}=x.(ab)=(x.a).b=(x^a)^b$.  For example $(12)(23)=(132)$ as we "read from the left".
\item $G$ acts on itself by right multiplication.  Let $X=G$ and $x.a:=xa$.  Then for $x, a, b\in G$ we have $(x.a).b=(xa).b=(xa)b=x(ab)=x.(ab)$ so $(A1)$ is satisfied.  Also, $x.1_G=x1_G=x$ so $(A2)$ is satisfied.
\item If we let $X=G$ and $x.a=ax$ then $(A2)$ is clearly satisfied.  For $x, a, b\in G$, $(x.a).b=(ax).b=b(ax)=(ba)x=x.(ba)\neq x.(ab)$.  In this case $(A2)$ is not satisfied, so left multiplication is not an action.\\
    However, we can modify this to make it work, as $G$ acts on $X=G$ by \textbf{inverse} left multiplication, where $x.a=a^{-1}x$.  Again, $(A2)$ is clearly satisfied.  Recall that $(ab)^{-1}=b^{-1}a^{-1}$ for any $a, b\in G$.  Hence, for $x, a, b\in G$ we have $(x.a).b=(a^{-1}x).b=b^{-1}(a^{-1}x)=(b^{-1}a^{-1})x=(ab)^{-1}x=x.(ab)$ so $(A1)$ is satisfied and we have an action.
\item $G$ acts on both sides of $X=G$ by \emph{conjugation}, i.e. $x.a=a^{-1}xa$.  To see this, let $x, a, b\in G$.  Then $(x.a).b=(a^{-1}xa).b=b^{-1}(a^{-1}xa)b=(b^{-1}a^{-1})x(ab)=(ab)^{-1}x(ab)=x.(ab)$ so $(A1)$ is satisfied.  Also, $x.1=1x1=x$ so $(A2)$ is satisfied.
\item $C_2$, the cyclic group consisting of $1$ and $a$, with $a^2=1$, acts on the divisors of $n\in\mathbb{N}$ by $d.a=n/d$.  We have $d.a^2=(d.a).a=(n/d).a=n/(n/d)=d$.
\end{enumerate}
\end{example}

\begin{remark}
Notice that an action needs no inverses, so action is a monoid concept.  If an element $a\in G$ has an inverse $a^{-1}$ then we have $(x.a).a^{-1}=x.(aa^{-1})=x.1=x$.
\end{remark}

\begin{definition}
A \emph{$G$-set} is a finite set $X$ together with an action of $G$ on $X$.  $G$-sets are permutation representations.
\end{definition}

\begin{theorem}[Action Theorem]
Suppose $X$ is a $G$-set.  Then $\Phi:G\rightarrow Sym(X)$, $a\mapsto\tilde{a}=(x\mapsto x.a)\in End(X)$ is a permutation representation of $G$ (on $X$).  Conversely, if $X$ is a finite set and $f:G\rightarrow Sym(X)$ is a permutation representation then $X$ is a $G$-set with $a.x=x^{f(a)}$ for all $x\in X$, $a\in G$.

\begin{proof}
To show that $\Phi:G\rightarrow Sym(X)$, $a\mapsto\tilde{a}=(x\mapsto x.a)\in End(X)$ is a permutation representation of $G$, we need to show that it is a homomorphism.  Let $a, b\in G$, then $\Phi(ab)=\widetilde{ab}=x\mapsto x.(ab)=x\mapsto(x.a).b=(x\mapsto x.a)\cdot(x\mapsto x.b)=\tilde{a}\cdot\tilde{b}=\Phi(a)\Phi(b)$, so $\Phi:G\rightarrow Sym(x)$ is a homomorphism and therefore a permutation representation.\\
Conversely, we need to show that if $f:G\rightarrow Sym(X)$ is a permutation representation, then $X\times G\rightarrow X$, $x.a\mapsto x^{f(a)}$ is an action.  Let $x\in X$ and $a, b\in G$.  Then $(x.a).b=(x^{f(a)}).b=(x^{f(a)})^{f(b)}=x^{f(a)f(b)}=x^{f(ab)}=x.(ab)$ so $(A1)$ is satisfied.  Also, $x.1=x^{f(1)}=x^1=x$ so $(A2)$ is satisfied.  Therefore we have an action, so $X$ is a $G$-set.
\end{proof}
\end{theorem}

\begin{example}
A set $X$ is trivially a $G$-set with $x.a=x$ for $x\in X$ and $a\in G$.
\end{example}

\begin{definition}
Let $G$ and $H$ be monoids. A map $f: G \rightarrow H$ is a homomorphism of monoids if:
\begin{enumerate}
\item $f(ab) = f(a)f(b)$ for all $a, b \in G$
\item $f(1_{G}) = 1_{H}$
\end{enumerate}
\end{definition}

\begin{remark}
If $H$ is a group, $a \in G$ then $f(a) \in H$ has an inverse $f(a)^{-1}$.

$1_{h} = f(a)f(a)^{-1} = f( 1_{G} a )f(a)^{-1} = f(1_{G})f(a)f(a)^{-1} = f(1_{G}) 1_{H} = f(1_{G}).$
Also $f(a^{-1}) = f(a)^{-1}$.
\end{remark}

\begin{example}
Let $H \leq G$ be a subgroup.
$G / H = \left\{ Ha : a \in G \right\}$ the set of all right cosets of $H$ in $G$.
$G$ acts on $G / H$. $Ha.b := H(ab)$, $b \in G$. $Ha = \left\{ ha : h \in H \right\}$. 
\end{example}

\begin{center}
\textbf{New actions from old}
\end{center}
Suppose $X$ is a $G$-set then:
\begin{itemize}
\item $G$ acts on $X^{k}: \left( x_{1}, \ldots, x_{k} \right) \cdot a := \left(  x_{1} a, \ldots, x_{k} a \right)$, $(A1)$ and $(A2)$.
\item $G$ acts on $2^{X} : Y \subseteq X$, $Ya := \left\{ xa : x \in Y \right\}.$
\end{itemize}

Now suppose $X$ and $Y$ are both $G$-sets.
\begin{itemize}
\item $G$ acts on $X \times Y: \left( x, y \right) \cdot a := \left( x \cdot a, y \cdot a \right) $
\item $G$ acts on $X \sqcup Y$
\end {itemize}

$z \cdot a:=
\begin{cases}
z \cdot_{x} a & \mbox{if } z \in X\\
z \cdot_{y} a & \mbox{if } z \in Y
\end{cases}$

This gives a notion of decomposability.

\begin{center}
\textbf{Orbits and Stabilizers}
\end{center}

\begin{definition}
Let $X$ be a $G$-set. $G_{x} = \left\{ a \in G : x \cdot a = x \right\}$ is the \emph{stabilizer} of $x$.
$xG = \left\{ x\cdot a : a \in G \right\}$ is the \emph{orbit} of $x$.
\end{definition}

\begin{theorem}
\begin{enumerate}
\item $G_{x}$ is a subgroup of $G$ \left( products and inverses of stabilizing elements do stabilize \right). 
\item $X / G = \left\{ x \cdot G : x \in X \right\}$ is a partition of $X$ into $G$-sets.
\item $x \cdot G \cong_{G} G / G_{x}$
\end{enumerate}
\end{theorem}

\begin{center}
\textbf{G-maps, isomorphisms of G-sets and similar actions}
\end{center}

\begin{definition}
Let $X$, $Y$ be $G$-sets. A map $f : X \rightarrow Y$ is a \emph{$G$-map} if $f(x \cdot a ) = f(x) \cdot a$ for all $a \in G$ and $x \in X$.
\end{definition}

\begin{example}
$D_{12}$ acts as the symmetries of a a regular hexagon with vertices labelled from 0 to 5.
Let $X = \left\{ 0, 1, 2, 3, 4, 5 \right\}$ and $Y = \left\{ \left\{0, 3 \right\}, \left\{ 1, 4 \right\}, \left\{ 2, 5 \right\} \right\}$
Define a map $f: X \rightarrow Y$ by $x \mapsto [ x ]$
$f( x \cdot a ) = [ x \cdot a ] = [ x ] \cdot a = f( x ) \cdot a$
Thus $f$ is a $G$-map.
\end{example}

 If $f$ is a bijection we call it a $G$-isomorphism. $X$ and $Y$ are isomorphic as $G$-sets \left( $X \cong_{G} Y$ \right) if there exists a $G$-isomorphism $f: X \rightarrow Y$.

\begin{itemize}
\item A $G$-set $X$ is called \emph transitive if $X = x \cdot G$, for some/all $x \in X$.
\item Every $G$-set is a disjoint union of transitive $G$-sets.
\item Every transitive $G$-set is similar to $G / H$ for some subgroup $H$ of $G$.
\end{itemize}


  
 

\nocite{*}

%%%%%%%%%%%%%%%%%%%%%%%%%%%%%%%%%%%%%%%%%%%%%%%%%%%%%%%%%%%%%%%%%%%%%%%%%%%%%
\bibliographystyle{amsplain}
\bibliography{comrepth}

%%%%%%%%%%%%%%%%%%%%%%%%%%%%%%%%%%%%%%%%%%%%%%%%%%%%%%%%%%%%%%%%%%%%%%%%%%%%%
\end{document}
